\documentclass[11pt,twoside,a4paper]{article}
\usepackage{amssymb}
\usepackage{amsmath}


\newcommand{\BIN}{\begin{bmatrix}}
\newcommand{\BOUT}{\end{bmatrix}}
\newcommand{\calR}{\mathcal{R}}
\newcommand{\calE}{\mathcal{E}}
\newcommand{\repr}{\cong}
\newcommand{\dpartial}[2]{\frac{\partial{#1}}{\partial{#2}}}
\newcommand{\ddpartial}[2]{\frac{\partial^2{#1}}{\partial{#2}^2}}

\newcommand{\aRb}{\ \leftidx{^A}R_B}
\newcommand{\aMb}{\ \leftidx{^A}M_B}
\newcommand{\amb}{\ \leftidx{^A}m_B}
\newcommand{\apb}{{\ \leftidx{^A}{AB}{}}}
\newcommand{\aXb}{\ \leftidx{^A}X_B}

\newcommand{\bRa}{\ \leftidx{^B}R_A}
\newcommand{\bMa}{\ \leftidx{^B}M_A}
\newcommand{\bma}{\ \leftidx{^B}m_A}
\newcommand{\bpa}{\ \leftidx{^B}{BA}{}}
\newcommand{\bXa}{\ \leftidx{^B}X_A}

\newcommand{\ap}{\ \leftidx{^A}p}
\newcommand{\bp}{\ \leftidx{^B}p}

\newcommand{\afs}{\ \leftidx{^A}\phi}
\newcommand{\bfs}{\ \leftidx{^B}\phi}
\newcommand{\af}{\ \leftidx{^A}f}
\renewcommand{\bf}{\ \leftidx{^B}f}
\newcommand{\an}{\ \leftidx{^A}\tau}
\newcommand{\bn}{\ \leftidx{^B}\tau}

\newcommand{\avs}{\ \leftidx{^A}\nu}
\newcommand{\bvs}{\ \leftidx{^B}\nu}
\newcommand{\w}{\omega}
\newcommand{\av}{\ \leftidx{^A}v}
\newcommand{\bv}{\ \leftidx{^B}v}
\newcommand{\aw}{\ \leftidx{^A}\w}
\newcommand{\bw}{\ \leftidx{^B}\w}

\newcommand{\aI}{\ \leftidx{^A}I}
\newcommand{\bI}{\ \leftidx{^B}I}
\newcommand{\cI}{\ \leftidx{^C}I}
\newcommand{\aY}{\ \leftidx{^A}Y}
\newcommand{\bY}{\ \leftidx{^B}Y}
\newcommand{\cY}{\ \leftidx{^c}Y}
\newcommand{\aXc}{\ \leftidx{^A}X_C}
\newcommand{\aMc}{\ \leftidx{^A}M_C}
\newcommand{\aRc}{\ \leftidx{^A}R_C}
\newcommand{\apc}{\ \leftidx{^A}{AC}{}}
\newcommand{\bXc}{\ \leftidx{^B}X_C}
\newcommand{\bRc}{\ \leftidx{^B}R_C}
\newcommand{\bMc}{\ \leftidx{^B}M_C}
\newcommand{\bpc}{\ \leftidx{^B}{BC}{}}

\usepackage{leftidx}

\begin{document}
\title{SE(3) operations}
\author{N. Mansard}
\date{}
\maketitle

\section{Rigid transformation}
$$m : p \in \calE(3) \rightarrow m(p) \in E(3)$$
Transformation from B to A:
$$\amb : \bp \in \calR^3 \repr \calE(3) \ \rightarrow\ \ap = \amb(\bp) = \aMb\ \bp$$
$$ \ap = \aRb \bp +  \apb$$
$$\aMb = \BIN \aRb & \apb \\ 0 & 1 \BOUT $$
Transformation from A to B:
$$\bp = \aRb^T \ap + \bpa, \quad\textrm{with }\bpa = - \aRb^T \apb$$
$$\bMa = \BIN \aRb^T & - \aRb^T \apb \\ 0 & 1 \BOUT $$
For Featherstone, $E = \bRa =\aRb^T$ and $r = \apb$. Then:
$$\bMa = \BIN \bRa & -\bRa \apb \\ 0 & 1 \BOUT = \BIN E & -E r \\ 0 & 1 \BOUT $$
$$\aMb = \BIN \bRa^T & \apb \\ 0 & 1 \BOUT = \BIN E^T & r \\ 0 & 1 \BOUT $$

\section{Composition}
$$ \aMb \bMc = \BIN \aRb \bRc & \apb +  \aRb \bpc \\ 0 & 1 \BOUT $$
$$ \aMb^{-1} \aMc = \BIN \aRb^T \aRc & \aRb^T (\apc - \apb) \\ 0 & 1 \BOUT $$



\section{Motion Application}
$$\avs = \BIN \av \\ \aw \BOUT$$
$$\bvs = \bXa\avs$$
$$ \aXb =  \BIN \aRb & \apb_\times \aRb \\ 0 & \aRb \BOUT $$
$$ \aXb^{-1} = \bXa =  \BIN \aRb^T & -\aRb^T \apb_\times \\ 0 & \aRb^T \BOUT $$
For Featherstone, $E = \bRa =\aRb^T$ and $r = \apb$. Then:
$$ \bXa = \BIN \bRa & - \bRa \apb_\times \\ 0 & \bRa \BOUT = \BIN E & -E r_\times \\ 0 & E \BOUT$$
$$ \aXb = \BIN \bRa^T & \apb_\times \bRa^T \\ 0 & \bRa^T \BOUT = \BIN E^T & r_\times E^T \\ 0 & E^T \BOUT$$ 

\section{Force Application}
$$\afs = \BIN \af \\ \an \BOUT$$
$$\bfs = \bXa^* \afs$$
For any $\phi,\nu$, $\phi\dot\nu = \afs^T \avs = \bfs^T \bvs$ and then:
$$\aXb^* = \aXb^{-T} = \BIN \aRb & 0 \\ \apb_\times \aRb & \aRb \BOUT$$
(because $\apb_\times^T = - \apb_\times$).
$$\aXb^{-*} = \bXa^* = \BIN \aRb^T & 0 \\ -\aRb^T \apb_\times  & \aRb^T \BOUT$$
For Featherstone, $E = \bRa =\aRb^T$ and $r = \apb$. Then:
$$\bXa^* = \BIN \bRa & 0 \\ -\bRa \apb_\times & \bRa \BOUT = \BIN E & 0 \\ - E r_\times & E \BOUT $$
$$\aXb^* = \BIN \bRa^T & 0 \\  \apb_\times \bRa^T & \bRa^T \BOUT = \BIN E^T & 0 \\ r_\times E^T & E^T \BOUT $$

\section{Inertia}
\subsection{Inertia application}

$$\aY: \avs \rightarrow \afs = \aY \avs$$

Coordinate transform:
$$\bY = \bXa^{*} \aY \bXa^{-1}$$
since: 
$$\bfs = \bXa^* \bfs = \bXa^* \aI \aXb \bvs$$
Cannonical form. The inertia about the center of mass $c$ is:
$$\cY = \BIN m & 0 \\ 0 & \cI \BOUT$$
Expressed in any non-centered coordinate system $A$:
$$\aY = \aXc^* \cI \aXc^{-1} = \BIN m & m\ ^AAC_\times^T \\  m\ ^AAC_\times & \aI + m \apc_\times \apc\times^T \BOUT $$
Changing the coordinates system from $B$ to $A$:
$$\aY = \aXb^* \bXc^* \cI \bXc^{-1} \aXb^{-1} $$
$$ = \BIN m & m [\apb + \aRb \bpc]_\times^T \\  m [\apb + \aRb \bpc]_\times & \aRb \bI \aRb^T - m [\apb + \aRb \bpc]_\times^2 \BOUT$$
Representing the spatial inertia in $B$ by the triplet $(m,\bpc,\bI)$, the expression in $A$ is:
$$ \amb: \bY = (m,\bpc,\bI) \rightarrow \aY = (m,\apb+\aRb \bpc,\aRb \bI \aRb^T)$$
Similarly, the inverse action is:
$$ \amb^{-1}: \aY \rightarrow \bY = (m,\aRb^T(^AAC-\apb),\aRb^T\aI \aRb) $$

Motion-to-force map:
$$ Y \nu = \BIN m & mc_\times^T \\ mc_\times & I+mc_\times c_\times^T \BOUT \BIN v \\ \omega \BOUT
 = \BIN m v - mc \times \omega \\ mc \times v + I \omega - mc \times ( c\times \omega) \BOUT$$

Nota: the square of the cross product is:
$$\BIN x\\y\\z\BOUT_ \times^2 = \BIN 0&-z&y \\ z&0&-x \\ -y&x&0 \BOUT^2 = \BIN -y^2-z^2&xy&xz \\ xy&-x^2-z^2&yz \\ xz&yz&-x^2-y^2 \BOUT$$
There is no computational interest in using it.

\subsection{Inertia addition}

$$ Y_p = \BIN m_p &  m_p  p_\times^T \\ m_p p_\times &  I_p + m_p  p_\times p_\times^T \BOUT$$
$$ Y_q = \BIN m_q &  m_q  q_\times^T \\ m_q q_\times &  I_q + m_q  q_\times q_\times^T \BOUT$$




\section{Cross products}

Motion-motion product:
$$\nu_1 \times \nu_2 = \BIN v_1\\\omega_1\BOUT \times \BIN v_2\\\omega_2\BOUT = \BIN  v_1 \times \omega_2 + \omega_1 \times v_2 \\ \omega_1 \times \omega_2 \BOUT $$
Motion-force product:
$$\nu \times \phi =  \BIN v\\\omega\BOUT \times \BIN f\\ \tau \BOUT = \BIN  \omega \times f \\ \omega \times \tau + v \times f \BOUT $$
A special form of the motion-force product is often used:
\begin{align*}\nu \times (Y \nu) &= \nu \times \BIN mv - mc\times \omega \\ mc\times v + I \omega - mc\times(c\times \omega) \BOUT \\&= \BIN m \omega\times v - \omega\times(mc\times \omega) \\ \omega \times ( mc \times v) + \omega \times (I\omega) -\omega \times(c \times( mc\times \omega)) -v\times(mc \times \omega)\BOUT\end{align*}
Setting $\beta=mc \times \omega$, this product can be written:
$$\nu \times (Y \nu) = \BIN \omega \times (m v - \beta) \\ \omega \times( c \times (mv-\beta)+I\omega) - v \times \beta \BOUT$$
This last form cost five $\times$, four $+$ and one $3\times3$ matrix-vector multiplication.

\section{Joint}

We denote by $1$ the coordinate system attached to the parent (predecessor) body at the joint input, ad by $2$ the coordinate system attached to the (child) successor body at the joint output. We neglect the possible time variation of the joint model (ie the bias velocity $\sigma = \nu(q,0)$ is null).

The joint geometry is expressed by the rigid transformation from the input to the ouput, parametrized by the joint coordinate system $q \in \mathcal{Q}$:
$$ ^2m_1 \repr \ ^2M_1(q)$$

The joint velocity (i.e. the velocity of the child wrt. the parent in the child coordinate system) is:
$$^2\nu_{12} = \nu_J(q,v_q) = \ ^2S(q) v_q $$
where $^2S$ is the joint Jacobian (or constraint matrix) that define the motion subspace allowed by the joint, and $v_q$ is the joint coordinate velocity (i.e. an element of the Lie algebra associated with the joint coordinate manifold), which would be $v_q=\dot q$ when $\dot q$ exists.

The joint acceleration is:
$$^2\alpha_{12} = S \dot v_q + c_J + \ ^2\nu_{1} \times \ ^2\nu_{12}$$
where $c_J = \sum_{i=1}^{n_q} \dpartial{S}{q_i} \dot q_i$ (null in the usual cases) and $^2\nu_{1}$ is the velocity of the parent body with respect to an absolute (Galilean) coordinate system\footnote{The abosulte velocity $\nu_{1}$ is also the relative velocity wrt. the Galilean coordinate system $\Omega$. The exhaustive notation should be $\nu_{\Omega1}$ but $\nu_1$ is prefered for simplicity.}.

The joint calculations take as input the joint position $q$ and velocity $v_q$ and should output $^2M_1$, $^2\nu_{12}$ and $^2c$ (this last vector being often a trivial $0_6$ vector). In addition, the joint model should store the position of the joint input in the central coordinate system of the previous joint $^0m_1$ which is a constant value.

The joint integrator computes the exponential map associated with the joint manifold. The function inputs are the initial position $q_0$, the velocity $v_q$  and the length of the integration interval $t$. It computes $q_t$ as:
$$ q_t = q_0 + \int_0^t v_q dt$$
For the simple vectorial case where $v_q=\dot q$, we have $q_t=q_0 + t v_q$. Written in the more general case of a Lie groupe, we have $q_t = q_0 exp(t v_q)$ where $exp$ denotes the exponential map (i.e. integration of a constant vector field from the Lie algebra into the Lie group). This integration only consider first order explicit Euler. More general integrators (e.g. Runge-Kutta in Lie groupes) remains to be written. Adequate references are welcome.

\section{RNEA}

\subsection{Initialization} 
$$^0\nu_0 = 0 ; \ ^0\alpha_0 = -g$$

In the following, the coordinate system $i$ is attached to the output of the joint (child body), while $lambda(i)$ is the central coordinate system attached to the parent joint. The coordinated system associated with the joint input is denoted by $i_0$. The constant rigid transformation from $\lambda(i)$ to the joint input is then $^{\lambda(i)}M_{i_0}$.


\subsection{Forward loop} 
For each joint $i$, update the joint calculation $\mathbf j_i$.calc($q,v_q$). This compute $\mathbf{j}.M = \ ^{\lambda(i)}M_{i_0}(q)$, $\mathbf{j}.\nu = \ ^i\nu_{{\lambda(i)}i}(q,v_q)$, $\mathbf{j}.S = \ ^iS(q)$  and $\mathbf{j}.c = \sum_{k=1}^{n_q} \dpartial{^iS}{q_k} \dot q_k$. Attached to the joint is also its placement in body $\lambda(i)$ denoted by $\mathbf{j}.M_0 =\ ^{\lambda(i)}M_{i_0}$. Then:

$$^{\lambda(i)}M_i = \mathbf{j}.M_0 \ \mathbf{j}.M $$
$$^0M_i = \ ^0M_{\lambda(i)} \ ^{\lambda(i)}M_i$$
$$^i\nu_{i}= \ ^{\lambda(i)}X_i^{-1} \ ^{\lambda(i)}\nu_{{\lambda(i)}} + \mathbf{j}.\nu$$
$$^i\alpha_{i}= \ ^{\lambda(i)}X_i^{-1} \  ^{\lambda(i)}\alpha_{{\lambda(i)}} + \mathbf{j}.S \dot v_q + \mathbf{j}.c +  \ ^i\nu_{i} \times  \mathbf{j}.\nu$$
$$^i\phi_i= \ ^iY_i \ ^i\alpha_i + \ ^i\nu_i \times \ ^iY_i \ ^i\nu_i - \ ^0X_i^{-*}\ ^0\phi_i^{ext}$$

\subsection{Backward loop} 
For each joint $i$ from leaf to root, do:
$$\tau_i = \mathbf{j}.S^T \ ^i\phi_i$$
$$^{\lambda(i)}\phi_{\lambda(i)} \ +\!\!= \ ^{\lambda(i)}X_i^{*} \ ^i\phi_i$$

\subsection{Nota}
It is more efficient to apply $X^{-1}$ than $X$. Similarly, it is more efficient to apply $X^{-*}$ than $X^*$. Therefore, it is better to store the transformations $^{\lambda(i)}m_i$ and $^0m_i$ than $^im_{\lambda(i)}$ and $^im_0$.


\end{document}
